\chapter*{Abstract}
This is the space reserved for the abstract of your report. The abstract is a summary of the report, so it is a good idea to write after all other chapters. The abstract for a thesis at PoliTO must be shorter than 3500 chars, try to be compliant with this rule (no problem for an abstract that is a lot shorter than 3500 chars, since this is not a thesis).
Use short sentences, do not use over-complicated words. Try to be as clear as possible, do not make logical leaps in the text. Read your abstract several times and check if there is a logical connection from the beginning to the end. The abstract is supposed to draw the attention of the reader, your goal is to write an abstract that makes the reader wanting to read the entire report. Do not go too far into details; if you want to provide data, do it, but express it in a simple way (e.g., a single percentage in a sentence): do not bore the reader with data that he or she cannot understand yet. Organize the abstract into paragraphs: the paragraphs are always 3 to 5 lines long. In \LaTeX source file, go new line twice to start a new paragraph in the PDF. Do not use $\\$ to go new line, just press Enter. In the PDF, there will be no gap line, but the text will go new line and a Tab will be inserted. This is the correct way to indent a paragraph, please do not change it. Do not put words in \textbf{bold} here: for emphasis, use \emph{italic}. Do not use citations here: they are not allowed in the abstract. Footnotes and links are not allowed as well. DO NOT EVER USE ENGLISH SHORT FORMS (i.e., isn't, aren't, don't, etc.).
Take a look at the following links about how to write an Abstract:
\begin{itemize}
\item \url{https://writing.wisc.edu/handbook/assignments/writing-an-abstract-for-your-research-paper/}
\item \url{https://www.anu.edu.au/students/academic-skills/research-writing/journal-article-writing/writing-an-abstract}
\end{itemize}
Search on Google if you need more info.