\chapter*{Abstract}
%This is the space reserved for the abstract of your report. The abstract is a summary of the report, so it is a good idea to write after all other chapters. The abstract for a thesis at PoliTO must be shorter than 3500 chars, try to be compliant with this rule (no problem for an abstract that is a lot shorter than 3500 chars, since this is not a thesis).
%Use short sentences, do not use over-complicated words. Try to be as clear as possible, do not make logical leaps in the text. Read your abstract several times and check if there is a logical connection from the beginning to the end. The abstract is supposed to draw the attention of the reader, your goal is to write an abstract that makes the reader wanting to read the entire report. Do not go too far into details; if you want to provide data, do it, but express it in a simple way (e.g., a single percentage in a sentence): do not bore the reader with data that he or she cannot understand yet. Organize the abstract into paragraphs: the paragraphs are always 3 to 5 lines long. In \LaTeX source file, go new line twice to start a new paragraph in the PDF. Do not use $\\$ to go new line, just press Enter. In the PDF, there will be no gap line, but the text will go new line and a Tab will be inserted. This is the correct way to indent a paragraph, please do not change it. Do not put words in \textbf{bold} here: for emphasis, use \emph{italic}. Do not use citations here: they are not allowed in the abstract. Footnotes and links are not allowed as well. DO NOT EVER USE ENGLISH SHORT FORMS (i.e., isn't, aren't, don't, etc.).
%Take a look at the following links about how to write an Abstract:
%\begin{itemize}
%\item \url{https://writing.wisc.edu/handbook/assignments/writing-an-abstract-for-your-research-paper/}
%\item \url{https://www.anu.edu.au/students/academic-skills/research-writing/journal-article-writing/writing-an-abstract}
%\end{itemize}
%Search on Google if you need more info.

%\emph{Background.} 
Recently, the amount of small electronic devices embedded in larger systems has grown exponentially. These devices play an essential role in almost all domains in everyday life, such as health care, houses and cities  automation, wearable devices, autonomous driving, industrial and farming applications. Unfortunately, this enormous growth goes hand in hand with the growth of counterfeit devices, thus giving rise to privacy and security concerns. 

To counteract this problem, an interesting solution is the use of PUFs (Physical Unclonable Functions). PUFs use instance-specific features of a device to provide a unique way to identify that device and therefore allowing a secure authentication method.

%\emph{Aim.} 
The goal of this project is to develop a SRAM based PUF for the SEcube\textsuperscript{TM}, a single-chip integratable device, to provide a secure authentication method between a host and the device.

%\emph{Approach.} 
The idea behind this authentication procedure is that whenever the host machine tries to connect to a device, a challenge-response mechanism takes place. The first time this connection is established, the device sends the host a list of values that the host will store for future reference. In later connections, the host asks the device to provide a specific response and, if the device answers correctly, the connection is approved, otherwise it is interrupted.

%\emph{Results.} The code developed in this project provides a considerably accurate SRAM PUF authentication mechanism. As expected, 100\% accuracy is unfeasible to achieve since the cells of an SRAM can assume slightly different values at each power on cycle due to empirical factors, such as temperature and aging. In order to reduce such errors, the hamming distance error-correcting method has been used.

%\emph{Conclusions.} 
The SRAM PUF developed in this project provides a way to securely authenticate a device when a connection to it must be established. %Possible improvements could be to encrypt the responses sent by the device and to delete a response once it has been used to increase the level of security of the connection between host and device.