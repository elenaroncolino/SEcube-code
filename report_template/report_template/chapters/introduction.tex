\chapter{Introduction}
% DELETE THE TEXT BELOW
%In this first chapter we expect you to introduce the project explaining %what the project is about, what is the final goal, what are the topics %tackled by the project, etc.\newline The introduction must not include %any low-level detail about the project, avoid sentences written like: we %did this, then this, then this, etc.\newline It is strongly suggested to %avoid expressions like `We think`, `We did`, etc...it is better to use %impersonal expressions such as: `It is clear that`, `It is possible %that`, `... something ... has been implemented/analyzed/etc.` (instead %of `we did, we implemented, we analyzed`).\newline In the introduction %you should give to the reader enough information to understand what is %going to be explained in the remainder of the report (basically, %expanding some concept you mentioned in the Abstract) without giving %away too many information that would make the introduction too long and %boring.\newline Feel free to organize the introduction in multiple %sections and subsections, depending on how much content you want to put %into this chapter.

%Remember that the introduction is needed to make the reader understand %what kind of reading he or she will encounter. Be fluent and try not to %confuse him or her.
%The introduction must ALWAYS end with the following formula:
In the last years, the number of small electronic devices that can be connected with big computational units; this led to the necessity to have a secure way to identify this kind of device in order to avoid the possibility of substituting them with an attack.
The PUF (Physical Unclonable Function) is a challenge-response mechanism that can be used for secure authentication.
The particularity of the PUFs is that they depend on the physical characteristics of the device; this makes it quite impossible to replace them without knowing.




The remainder of the document is organized as follows. 


In Chapter 2, ...; 

In Chapter 3, ... 



%so that the reader can choose which chapters are worth skipping %according to the type of reading he or she has chosen.

