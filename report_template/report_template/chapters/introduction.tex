\chapter{Introduction}
% DELETE THE TEXT BELOW
%In this first chapter we expect you to introduce the project explaining %what the project is about, what is the final goal, what are the topics %tackled by the project, etc.\newline The introduction must not include %any low-level detail about the project, avoid sentences written like: we %did this, then this, then this, etc.\newline It is strongly suggested to %avoid expressions like `We think`, `We did`, etc...it is better to use %impersonal expressions such as: `It is clear that`, `It is possible %that`, `... something ... has been implemented/analyzed/etc.` (instead %of `we did, we implemented, we analyzed`).\newline In the introduction %you should give to the reader enough information to understand what is %going to be explained in the remainder of the report (basically, %expanding some concept you mentioned in the Abstract) without giving %away too many information that would make the introduction too long and %boring.\newline Feel free to organize the introduction in multiple %sections and subsections, depending on how much content you want to put %into this chapter.

%Remember that the introduction is needed to make the reader understand %what kind of reading he or she will encounter. Be fluent and try not to %confuse him or her.
%The introduction must ALWAYS end with the following formula:

In the last years, the number of small electronic devices that can be connected with big computational units grew exponentially. Embedded systems play a crucial role in fueling the growth of the Internet-of-Things (Iot) in the most diverse domains, such as health care, home automation and transportation. By the end of 2022 the number of IoT devices connected to the Internet is expected to reach the astonishing number of 14.4 billions \cite{IoT_state}. The ubiquitousness of such devices coupled with their ability to access potentially sensitive/confidential information has given rise to security and privacy concerns. An additional challenge is the growing number of counterfeit
components in these devices, resulting in serious reliability and financial implications.

Physical unclonable functions (PUFs) are a promising security primitive to help address these concerns.  PUFs extract secrets from physical characteristics of integrated circuits (ICs)  \cite{PUF_IEEE_Herder} and therefore require minimal or no additional hardware for their operation and are therefore cheaper than other solutions. The instance-specific nature of the secret provide a mean to uniquely identify and authenticate each device based on a challenge-response mechanism \cite{PUF_Sutar}.

The aim of this project is to design and develop a RAM based PUF for the SEcube\textsuperscript{TM}, a single-chip easily integratable device capable of hiding significant complexity behind a set of simple high-level APIs \cite{SEcube_Prinetto}.

\vspace{1.5em}
The remainder of the document is organized as follows:
\vspace{1em}

In Chapter 2, a brief background and state of the art of this topic is provided; 

In Chapter 3, an implementation  overview is presented;

In Chapter 4, implementation details are described;

In Chapter 5, results are listed;

In Chapter 6, conclusions and final observations are presented.

Appendix A describes how a demo of the implementation can be run.

Appendix B describes the APIs created for this project.




%so that the reader can choose which chapters are worth skipping %according to the type of reading he or she has chosen.

